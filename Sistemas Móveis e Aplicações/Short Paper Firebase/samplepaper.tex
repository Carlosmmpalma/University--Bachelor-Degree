\documentclass[runningheads]{llncs}

\usepackage{graphicx}

\begin{document}
%
\title{Desenvolvimento de Aplicativos Móveis com a Plataforma Firebase}
%
%\titlerunning{Abbreviated paper title}
% If the paper title is too long for the running head, you can set
% an abbreviated paper title here
%
\author{Carlos Palma\orcidID{46520}}

\institute{Universidade de Évora, Engenharia Informática, \boldsymbol{2022/2023}
\email{l46520@alunos.uevora.pt}}
%
\maketitle              % typeset the header of the contribution
%
\begin{abstract}
 Este short paper apresenta a plataforma Firebase da Google, que é amplamente utilizada para desenvolver aplicações móveis. O Firebase oferece recursos abrangentes para autenticação de utilizadores, armazenamento de dados, mensagens em tempo real e análise de dados, tornando-se uma solução completa para programadores que desejam criar aplicações móveis de alta qualidade. Neste artigo, exploramos os principais recursos do Firebase e como eles podem ser usados para desenvolver aplicações móveis eficientes e escaláveis. Além disso, discutimos as vantagens e desvantagens de usar o Firebase para o desenvolvimento de aplicações móveis. No final, apresentamos exemplos desenvolvidos com a plataforma Firebase para ilustrar os seus recursos e funcionalidades.

\keywords{Firebase  \and Google \and Aplicação móvel.}
\end{abstract}
%
%
%
\section{Introdução}
\paragraph{}
Firebase é uma plataforma de desenvolvimento de aplicações móveis e web que fornece um conjunto abrangente de ferramentas e serviços para ajudar os programadores a criar aplicações de alta qualidade com rapidez e facilidade. Desde o seu lançamento em 2011, o Firebase tem evoluído constantemente para fornecer uma plataforma ainda mais poderosa e flexível.
A plataforma é fornecida pela Google e possui uma ampla gama de recursos, incluindo armazenamento em nuvem, autenticação de utilizador, base de dados em tempo real, análise de aplicações, testes de aplicações, razões pelas quais é amplamente utilizada em todo o mundo.
Esta foi lançada inicialmente como um serviço de base de dados em tempo real, tendo ao longo dos anos evoluido para uma plataforma completa de desenvolvimento de aplicações. \cite{1}.


\section{Revisão da literatura}
\paragraph{}
Segundo Zhang et al. (2018), Firebase é uma plataforma popular para desenvolvimento de aplicações móveis pela sua facilidade de uso e a sua integração com outras ferramentas da Google, como o Google Cloud Platform. Os autores também destacam que esta é uma plataforma completa, com recursos que atendem às necessidades do programador.\cite{2}

Um estudo de caso realizado por Alharbi et al. (2021) demonstrou que o uso do Firebase pode ser benéfico para o desenvolvimento de aplicações móveis que precisam de uma infraestrutura robusta para gerir grandes quantidades de dados,facilitando o desenvolvimento e tornando o processo mais rápido e eficiente.

Num estudo realizado por Kaur (2022) este demonstrou que o uso do Firebase Auth deve ter em conta, em todas as etapas, as medidades de seguranças apropriadas pois qualquer erro pode levar a uma baixa segurança na aplicação móvel.\cite{3}



\section{Método}
\paragraph{}
O método utilizado para desenvolver este short paper consistiu numa revisão da literatura relacionada com a plataforma Firebase da Google e a sua utilização no desenvolvimento de aplicações móveis. A pesquisa foi realizada em bases de dados científicas como o Google Scholar e a ACM Digital Library, bem como em artigos de revistas e conferências especializadas na área, tendo utilizado as palavras: firebase, mobile app, autenticação e segurança. Foram identificados vários estudos que abordam as vantagens e desvantagens da utilização do Firebase em aplicações móveis.\cite{4}Também foram analisadas as medidas de segurança recomendadas para garantir a privacidade dos utilizadores em aplicações que utilizam o Firebase Auth.\cite{5}

\section{Resultados}
\paragraph{}
Após o estudo das funcionalidades de backend e segurança do Firebase, podemos afirmar que a plataforma apresenta diversas capacidades importantes para o desenvolvimento de aplicações seguras e escaláveis. Algumas das principais conclusões do nosso estudo incluem:
\begin{itemize}

      \item Análise de dados: A plantaforma tem uma excelente solução de análise de dados "Firebase Analytics" que fornece ao programador detalhes sobre a forma como o utilizador interage com a aplicação.\cite{6}
      
      \item Armazenamento de dados em nuvem: A plantaforma Firebase oferece um serviço de cloud "Firebase Cloud" que permite ao utilizador armazenar os mais variados tipos de dados em nuvem.\cite{6}
      
      \item Armazenamento de dados em tempo real: O Firebase Realtime Database é uma poderosa ferramenta de backend que permite o armazenamento e sincronização de dados em tempo real. Isso possibilita a atualização imediata dos dados em diferentes dispositivos, sem a necessidade de atualizações manuais. Além disso, a plataforma oferece recursos avançados de consulta e filtragem de dados, o que facilita a obtenção de informações relevantes de forma rápida e eficiente.\cite{6}

      \item Autenticação e autorização de utilizadores: O Firebase Authentication é uma ferramenta de autenticação e autorização de utilizadores que permite o controlo de acesso aos dados e funcionalidades da aplicação. Assim, é possível implementar diferentes métodos de autenticação, como email e password, Google, Facebook, entre outros, com isto, apenas utilizadores autorizados têm acesso às informações sensíveis da aplicação, aumentando a segurança do sistema como um todo. \cite{7}

      \item Segurança de dados: A plataforma Firebase oferece diversos recursos de segurança de dados para proteger as informações sensíveis armazenadas na aplicação. Além das funcionalidades de autenticação e autorização de utilizadores, a plataforma também conta com recursos avançados de criptografia e validação de regras de segurança, desta forma é garantindo que apenas utilizadores autorizados possam aceder ou alterar as informações. \cite{6}

\end{itemize}

\section{Discussão}
\paragraph{}
Apesar dos recursos avançados da plataforma, é importante destacar que a utilização destes recursos requer um conhecimento técnico adequado por parte dos programadores. Sem esse conhecimento, pode ser difícil implementar as funcionalidades com eficácia e obter os resultados pretendidos. Outro ponto importante a ser destacado é que, embora esta forneça recursos abrangentes de segurança de dados, ainda é responsabilidade do programador garantir que as regras de segurança e privacidade sejam seguidas corretamente. Portanto, a plataforma não é uma solução de segurança completa e isolada, mas sim uma ferramenta que deve ser utilizada em conjunto com outras medidas de segurança, como criptografia e autenticação. 
No entanto de forma geral esta plantaforma é sem duvida uma excelente opção para programadores de aplicações móveis tendo em conta todos os aspectos positivos mencionados ao longo deste short paper ao qual se alia a ampla utilização por grandes empresas tais como SHAZAM, VENMO e TRIVAGO, sendo estas apenas um pequeno exemplo.

\section{Conclusão}
\paragraph{}
Com base nos resultados obtidos nesta pesquisa, pode-se concluir que o Firebase é uma plataforma altamente robusta e abrangente para o desenvolvimento de aplicações móveis. As funcionalidades de backend e segurança de dados fornecidas pela plataforma satisfazem as principais necessidades dos programadores, permitindo a criação de aplicações modernas, escaláveis e seguras.

%
% ---- Bibliography ----
%
% BibTeX users should specify bibliography style 'splncs04'.
% References will then be sorted and formatted in the correct style.
%
% \bibliographystyle{splncs04}
% \bibliography{mybibliography}
%
\begin{thebibliography}{8}
\bibitem{1}
Moroney, Laurence, Anglin Moroney, and Anglin. Definitive Guide to Firebase. California: Apress, 2017. \doi{10.1007/978-1-4842-2943-9}

\bibitem{2}
Harty, J., Zhang, H., Wei, L., Pascarella, L., Aniche, M., & Shang, W. (2021, May). Logging practices with mobile analytics: An empirical study on firebase. In 2021 IEEE/ACM 8th International Conference on Mobile Software Engineering and Systems (MobileSoft) (pp. 56-60). IEEE\doi{10.1109/MobileSoft52590.2021.00013}

\bibitem{3}
Kaur, Ishpreet. "Mobile Cloud Computing: Using Firebase Auth." (2022).
\doi{10.47760/ijcsmc.2022.v11i04.008}

\bibitem{4}
Firebase - https://firebase.google.com/

\bibitem{5}
Firebase Auth - https://firebase.google.com/docs/auth

\bibitem{6}
Moroney, Laurence, and Laurence Moroney. "The firebase realtime database." The Definitive Guide to Firebase: Build Android Apps on Google's Mobile Platform (2017): 51-71.
\doi{10.1007/978-1-4842-2943-9_3}

\bibitem{7}
Khawas, Chunnu, and Pritam Shah. "Application of firebase in android app development-a study." International Journal of Computer Applications 179.46 (2018): 49-53. https://www.researchgate.net/profile/Chunnu-Khawas/publication/325791990_Application_of_Firebase_in_Android_App_Development-A_Study/links/5bab55ed45851574f7e6801e/Application-of-Firebase-in-Android-App-Development-A-Study.pdf

\end{thebibliography}
\end{document}
