\documentclass{article}
\usepackage[utf8]{inputenc}
\usepackage[portuguese]{babel}

\title{"Facial Recognition with AI"}
\author{Carlos Palma 46520, Gonçalo Veríssimo 48738}
\date{Março 2022}

\begin{document}
\maketitle
\begin{abstract}

\textbf{Reconhecimento facial através da Inteligêcia Artificial}.
\paragraph{}
Decidimos escolher este tema porque apesar de vivermos num mundo tecnológico, 
todos os dias surgem novas tecnologias, e, o aparecimento
da Inteligência Artificial levou ao desenvolvimento do reconhecimento facial. 
O reconhecimento facial é no fundo uma tecnologia baseada em Inteligência Artificial que identifica a identidade de uma pessoa a partir duma imagem da sua cara.
A recorrência à Inteligência artificial e consequentemente ao reconhecimento facial são usados em muitos países e hoje em dia é usado para solucionar problemas com autenticações móveis e 
desbloqueamento de aplicações. Como facto curioso e bom exemplo de reconhecimento facial,
foi criado um programa que resolve em segundos um enigma que durava 15 anos.
Há que ter em mente que até as forças policias e de segurança de todo o mundo testam sistemas de reconhecimento
facial como uma forma de identificar criminosos e fugitivos.
O desenvolvimento deste tema leva a uma melhor prevenção de fraudes, 
a uma maior produtividade e automatização de processos e
a uma verificação de identidade de modo a vivermos numa sociedade mais protegida e segura.

No nosso artigo abordaremos a evolução da Inteligêcia Artificial no reconhecimento facial e o quão importante é nos dias de hoje.

Nesta pesquisa bibliográfica utilizámos o Google Scholar como motor de busca. As palavras-chaves foram \textbf{AI e Facial Recognition}.
\end{abstract}

    \cite{a}
    \cite{b}
    \cite{c}
    \cite{d}
    \cite{e}
    \cite{f}
    \cite{g}
    
\bibliographystyle{acm}
\bibliography{biblioteca}
\end{document}
