\documentclass{article}
\usepackage[portuguese]{babel}
\usepackage[utf8]{inputenc}
\usepackage{url}
\usepackage{hyperref}
\title{"Facial Recognition with AI"}
\author{Carlos Palma 46520,Gonçalo Veríssimo 48738}
\date{Março 2022}

\begin{document}
\maketitle
\begin{abstract}

\textbf{Facial Recognition with AI} \textbf{Reconhecimento facial através da Inteligêcia Artificial}, "Os recentes avanços na área da Inteligência Artificial e big data possibilitaram o desenvolvimento de tecnologias de reconhecimento facial. Este artigo tem como objetivo investigar como a utilização de tais tecnologias pode gerar violações ao direito à privacidade e à proteção de dados. Conclui-se haver necessidade de desenvolvimento de tecnologias em consonância com os princípios dispostos nas legislações de proteção de dados, a fim de se garantir a salvaguarda desses direitos."

No nosso artigo abordamos a evolução da Inteligêcia Artificial no reconhecimento facial e o quanto é importante nos dias de hoje.

Nesta pesquisa bibliográfica utilizámos o Google Scholar como motor de busca. As palavras-chaves foram \textbf{AI e Facial Recognition}.

\end{abstract}

\cite{a}
\cite{b}
\cite{c}
\cite{d}
\cite{e}
\cite{f}
\cite{g}

\bibliographystyle{plain}
\bibliography{biblioteca}

\end{document}